\documentclass{article}
\usepackage{amsmath, listings, booktabs, graphicx}
\usepackage[margin=1in]{geometry}

\title{Fluxonic Cosmology: A Unified Framework for Space-Time, Electromagnetism, and Gravity}
\author{Tshuutheni Emvula and Independent Frontier Science Collaboration}
\date{February 20, 2025}

\begin{document}

\maketitle

\section*{Abstract}
We consolidate Fluxonic Cosmology, addressing electromagnetic completeness, gravitational mediation, and falsifiability. Building on Dewey B. Larson’s Reciprocal System Theory, we integrate a vector potential for EM interactions, a derived fluxonic stress-energy tensor for gravity, and an expansion model eliminating dark energy. Simulations validate soliton-driven field interactions, gravitational distortions, and cosmic structure formation, offering a testable alternative to \(\Lambda\)CDM cosmology.

\section{Introduction}
Fluxonic Cosmology unifies prior research, responding to critiques of theoretical and computational gaps across over 25 iterations. Rooted in Larson’s Reciprocal System Theory, where motion is fundamental and space-time are reciprocal, it refines:
\begin{itemize}
    \item Electromagnetic theory with vector potentials.
    \item Gravitational effects via a stress-energy tensor.
    \item Cosmic expansion without dark energy.
    \item Simulation-validated field interactions and structure formation.
\end{itemize}
This presents a falsifiable framework challenging traditional cosmology.

\section{Mathematical Formulation}

\subsection{Electromagnetic Model Refinement}
The fluxonic electromagnetic model uses a Lagrangian:
\begin{align}
    \mathcal{L}_{\text{fluxon-EM}} &= -\frac{1}{4} F_{\mu\nu} F^{\mu\nu} + \frac{1}{2} (\partial_\mu \phi \partial^\mu \phi) - V(\phi) - J^\mu A_\mu, \\
    F_{\mu\nu} &= \partial_\mu A_\nu - \partial_\nu A_\mu, \\
    V(\phi) &= \frac{m^2}{2} \phi^2 + \frac{g}{4} \phi^4,
\end{align}
yielding:
\begin{align}
    E &= -\nabla \phi - \frac{\partial A}{\partial t}, \\
    B &= \nabla \times A, \\
    \frac{\partial E}{\partial t} &= \nabla \times B - \frac{J}{\epsilon_0}, \\
    \frac{\partial B}{\partial t} &= -\nabla \times E,
\end{align}
where:
\begin{equation}
    J^\mu = q_\phi (\phi \partial^\mu \phi) - \sigma A^\mu,
\end{equation}
with \( q_\phi \) and \(\sigma\) as coupling constants. This resolves scalar-only limits and ensures soliton-EM consistency.

\subsection{Gravitational Stress-Energy Tensor}
Gravitational effects are modeled by:
\begin{equation}
    G_{\mu\nu} + \Lambda g_{\mu\nu} = 8\pi G (T^{\text{fluxon}}_{\mu\nu} + T^{\text{EM}}_{\mu\nu}),
\end{equation}
where:
\begin{align}
    T^{\text{fluxon}}_{\mu\nu} &= \rho u_\mu u_\nu + p (g_{\mu\nu} + u_\mu u_\nu), \\
    \rho &= \frac{1}{2} (\dot{\phi}^2 + (\nabla \phi)^2) + V(\phi), \\
    p &= \frac{1}{2} (\dot{\phi}^2 - (\nabla \phi)^2) - V(\phi), \\
    u_\mu &= \frac{\partial_\mu \phi}{\sqrt{\partial^\alpha \phi \partial_\alpha \phi}}, \\
    T^{\text{EM}}_{\mu\nu} &= F_{\mu\alpha} F_\nu{}^\alpha - \frac{1}{4} g_{\mu\nu} F_{\alpha\beta} F^{\alpha\beta}.
\end{align}
This derives \(\rho\) and \(p\) from \(\phi\), capturing solitonic gravitational mediation.

\subsection{Cosmic Expansion Model}
Cosmic expansion is driven by:
\begin{equation}
    a(t) = e^{H t}, \quad H = \sqrt{\frac{8 \pi G}{3} \left( \rho_{\text{fluxon}} + \rho_{\text{EM}} \right)},
\end{equation}
where \(\rho_{\text{fluxon}}\) and \(\rho_{\text{EM}}\) stem from \(\mathcal{L}_{\text{fluxon-EM}}\), eliminating dark energy needs.

\section{Computational Validation}
Simulations validate:
\begin{itemize}
    \item Stable EM filaments mediating \( E \) and \( B \)-fields (Figure 1).
    \item Metric distortions in high-density fluxonic regions (Figure 2).
    \item Filamentary cosmic networks (Figure 3).
    \item Soliton collision coherence (Figure 4).
\end{itemize}

\section{Experimental Falsifiability}
Proposed tests include:
\begin{itemize}
    \item Superfluid analogs using BECs with fluxonic interactions.
    \item Gravitational wave attenuation in dense fluxonic fields.
    \item EM field responses to solitonic interactions.
\end{itemize}

\section{Conclusion and Future Work}
This framework unifies:
\begin{itemize}
    \item A complete EM model with vector potentials.
    \item A derived gravitational tensor.
    \item An expansion model sans dark energy.
    \item Simulation-validated phenomena.
\end{itemize}
Future work:
\begin{itemize}
    \item Expand gravitational simulations to larger scales.
    \item Predict CMB signatures.
    \item Test experimentally.
\end{itemize}
Fluxonic Cosmology offers a testable alternative to \(\Lambda\)CDM.

\end{document}