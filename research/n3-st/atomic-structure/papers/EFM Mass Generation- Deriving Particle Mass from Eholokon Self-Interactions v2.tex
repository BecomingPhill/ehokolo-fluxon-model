\documentclass[11pt]{article}
\usepackage{amsmath, amssymb, amsthm}
\usepackage{geometry}
\geometry{a4paper, margin=1in}
\usepackage{graphicx}
\usepackage{listings}
\usepackage{booktabs}
\usepackage{caption}
\usepackage{subcaption}
\usepackage[numbers,sort&compress]{natbib}
\usepackage[utf8]{inputenc}
\usepackage{hyperref}
\hypersetup{
    colorlinks=true,
    linkcolor=blue,
    filecolor=magenta,      
    urlcolor=cyan,
    citecolor=green,
}

\lstset{
  language=Python,
  basicstyle=\footnotesize\ttfamily,
  breaklines=true,
  numbers=left,
  numberstyle=\tiny\color{gray}, % Smaller line numbers
  commentstyle=\color{gray},
  frame=single,
  keywordstyle=\color{blue},
  stringstyle=\color{red},
  showstringspaces=false,
  tabsize=2 % Reduce tab size
}

\raggedbottom
\Urlmuskip=0mu plus 2mu\relax
\hyphenation{Eho-loko Flux-on Har-monic-Den-sity Re-cip-rocal-Sys-tem Klein-Gor-don non-lin-ear eho-lo-kon}
\setlength{\parskip}{0.5\baselineskip}

\title{EFM Mass Generation: Deriving Particle Mass from Eholokon Self-Interactions}
\author{Tshuutheni Emvula\thanks{Independent Researcher, Team Lead, Independent Frontier Science Collaboration}}
\date{June 11, 2025} % Updated Date

\begin{document}

\maketitle

\begin{abstract}
The Standard Model (SM) attributes particle mass to the Higgs mechanism, an addition requiring a detectable boson. The Eholoko Fluxon Model (EFM) offers a contrasting paradigm, proposing that mass is an intrinsic, emergent property of stable, localized eholokon (solitonic) structures arising from the self-interactions of a single scalar field (\(\phi\)). This paper presents a direct computational derivation of particle-like mass within the EFM framework. Utilizing a definitive, high-resolution 3D Nonlinear Klein-Gordon (NLKG) simulation on a \(450^3\) grid, tuned to the EFM's S=T (resonant) state and the n'=1 Harmonic Density State, we demonstrate the formation of a stable, non-zero soliton from an initial Gaussian pulse. The final stable soliton possesses a quantifiable mass integral (\(\int|\phi|^2 dV \approx 4.95 \times 10^5\) sim. units), leading to an effective mass (\(M_{\text{eff,sim}} \approx 4.95 \times 10^3\) sim. units). Scaling this to the physical electron mass yields a fundamental EFM simulation mass unit of \(1.84 \times 10^{-34}\) kg/sim\_mass\_unit. The derived characteristic Gaussian width of this electron analogue is \( \sigma_{\text{EFM}} \approx 1.28 \times 10^{-10} \) m, approximately 53 times the electron's Compton wavelength, a distinct, falsifiable prediction of the model. An analysis of the fine structure constant (\(\alpha_{\text{em}}\)) indicates that the dimensionless charge coupling, \(q_{\text{sim}}\), must be calibrated to \( \approx 1.12 \) to match physical reality, identifying it as an order-unity parameter. Perturbation analysis reveals energy level gaps on the MeV/GeV scale, consistent with particle physics. This work provides strong, simulation-backed evidence for a Higgs-less mass generation mechanism within EFM, deriving particle mass directly from field dynamics.
\end{abstract}

\section{Introduction}
The origin of particle mass is a central question in fundamental physics. The Standard Model (SM) incorporates mass via the Higgs mechanism, invoking a scalar field to impart mass to elementary particles \citep{SMReviewPlaceholder}. While empirically successful, this mechanism introduces parameters not derived from first principles. The Eholoko Fluxon Model (EFM) offers an alternative, rooted in Dewey B. Larson's Reciprocal System Theory (RST) \citep{larson1959}, positing that all physical phenomena, including particle properties like mass, emerge from the dynamics of a single scalar eholokon field (\(\phi\)) \citep{emvula2025compendium_intro}. EFM operates within a framework of Harmonic Density States (HDS), discrete, stable average density levels (\(\rho_{n'} = \rho_{\text{ref}}/n'\)) for the \(\phi\) field, derived computationally from EFM's Nonlinear Klein-Gordon (NLKG) equation \citep{emvula2025efm_hds_validated}.

This paper focuses on deriving particle mass from first principles within EFM. We hypothesize that fundamental particles are stable, localized eholokon (soliton) structures, and their mass is an emergent property proportional to the integrated intensity of their field configuration. Specifically, we simulate the formation of an eholokon in the S=T (resonant, optical) state, operating within the n'=1 HDS (highest density), as an analogue for the electron. We present high-resolution 3D NLKG simulation results demonstrating the formation and stabilization of such a structure, calculate its effective mass and characteristic size, and discuss the implications for fundamental constants like the fine structure constant. This provides a computationally validated, deterministic alternative to the Higgs mechanism.

\section{Mathematical and Computational Framework}

\subsection{EFM Nonlinear Klein-Gordon Equation for S=T State}
The eholokon field \(\phi\) dynamics for mass generation in the S=T (resonant) state are modeled by:
\begin{equation}
\frac{\partial^2 \phi}{\partial t^2} - c_{\text{sim}}^2 \nabla^2 \phi + V'(\phi) = 0
\label{eq:nlkg_massgen}
\end{equation}
where \(c_{\text{sim}}=1.0\) (simulation units, with \(dx_{\text{sim}}=1\)). The derivative of the self-interaction potential \(V'(\phi)\), based on the parameters required for stable soliton formation in this state \citep{emvula2025dimensionless_params}, is:
\begin{equation}
V'(\phi) = m_{\text{sq}}^2 \phi + g_{\text{p}} \phi^3 + \eta_{\text{p}} \phi^5
\label{eq:potential_derivative}
\end{equation}
with coefficients \(m_{\text{sq}}^2 = 1.0\), \(g_{\text{p}} = -0.1\), and \(\eta_{\text{p}} = 0.01\) (dimensionless simulation units). The attractive cubic term ($g_p < 0$) is crucial for creating the potential well necessary for a stable, localized, non-zero field configuration.

\subsection{Emergent Mass Definition}
The effective mass (\(M_{\text{eff,sim}}\)) of a stable eholokon (\(\phi_0\)) is defined as being proportional to its integrated field intensity:
\begin{equation}
M_{\text{eff,sim}} = k_{\text{mc,sim}} \int |\phi_0|^2 dV_{\text{sim}}
\label{eq:mass_definition}
\end{equation}
where \(k_{\text{mc,sim}} = 0.01\) is the dimensionless mass coupling constant.

\subsection{Definitive Simulation Setup}
The simulation was performed using an optimized, GPU-centric PyTorch implementation on a Google Colab A100 GPU.
\begin{itemize}
    \item \textbf{Grid \& Resolution:} \(N=450^3\).
    \item \textbf{Physical Domain:} Box size \(L_{\text{phys}} = 50\) Å. Physical grid spacing \(dx_{\text{phys}} \approx 1.11 \times 10^{-11}\) m.
    \item \textbf{Simulation Units:} \(dx_{\text{sim}} = 1.0\), \(c_{\text{sim}} = 1.0\).
    \item \textbf{Time Integration:} Timestep \(dt_{\text{sim}}\) from a CFL factor of 0.025, yielding \(dt_{\text{sim}} = 0.025\). Total steps \(T_{\text{steps}} = 10,000\).
    \item \textbf{Initial Condition:} A Gaussian pulse \(\phi_{\text{initial}}(r) = A_0 \exp(-r^2/w_0^2)\) with \(A_0 = 12.0\) and width \(w_0 = 15.0 \cdot dx_{\text{sim}}\).
    \item \textbf{Boundary Conditions:} Absorbing boundaries (15\% width, damping strength 0.2) to allow excess energy to radiate away.
\end{itemize}

\section{Simulation Results and Analysis}

\subsection{Formation of a Stable Soliton}
The simulation was evolved for 10,000 timesteps. The system rapidly radiated away excess energy from the initial pulse, settling into a highly stable, localized eholokon structure, as shown in Figure \ref{fig:mass_gen_evolution}.

\begin{figure}[htbp]
    \centering
    \includegraphics[width=\textwidth]{Energy Evolution.png}
    \caption{Evolution metrics for the definitive N=450, T=10000 run: Max Amplitude \(\max(|\phi|)\) (left), Total Energy (center), and Mass Integral \(\int|\phi|^2dV\) (right). All three metrics show the system evolving from the initial high-energy pulse and settling into a stable, non-zero, finite-energy state characteristic of a soliton.}
    \label{fig:mass_gen_evolution}
\end{figure}

The final stable state has the following properties in simulation units:
\begin{itemize}
    \item \textbf{Stable Max Amplitude:} \(\max(|\phi|) \approx 9.90\)
    \item \textbf{Stable Total Energy:} \(\approx 4.21 \times 10^6\)
    \item \textbf{Stable Mass Integral:} \(\int|\phi|^2dV \approx 4.95 \times 10^5\)
\end{itemize}
Using Eq. \ref{eq:mass_definition}, the final effective mass is \(M_{\text{eff,sim}} = 0.01 \times 4.9505 \times 10^5 = 4.9505 \times 10^3\) sim. units. Visualizations of the final stable soliton (Figure \ref{fig:soliton_slices_zoomed}) show a highly localized and complex internal structure.

\begin{figure}[htbp]
    \centering
    \includegraphics[width=\textwidth]{Physical Scaling Slice.png}
    \caption{2D slices of the final stable eholokon field \(\phi\). The structure is highly localized and exhibits complex internal nodes, a hallmark of EFM's non-linear dynamics. Physical extent shown is \(\approx \pm 2.2\) Ångströms from the center.}
    \label{fig:soliton_slices_zoomed}
\end{figure}

\subsection{Physical Scaling and Soliton Size}
By identifying this stable soliton as the EFM electron analogue, we can derive the fundamental EFM simulation mass unit:
\[
\text{Mass Unit}_{\text{kg/sim}} = \frac{M_{\text{electron}}}{M_{\text{eff,sim}}} = \frac{9.109 \times 10^{-31} \text{ kg}}{4.9505 \times 10^3 \text{ sim. units}} \approx 1.840 \times 10^{-34} \text{ kg/sim\_mass\_unit}
\]
The soliton's size was characterized by fitting a Gaussian to a 1D profile of the final state (Figure \ref{fig:ehokolon_profile_fit}).
\begin{figure}[htbp]
    \centering
    \includegraphics[width=0.7\textwidth]{Gaussian Fit.png}
    \caption{1D profile of the final eholokon state and its Gaussian fit. The fit yields a characteristic width \(\sigma_{\text{sim}} \approx 11.56 \, dx_{\text{sim}}\).}
    \label{fig:ehokolon_profile_fit}
\end{figure}
The fit yielded \(\sigma_{\text{sim}} \approx 11.56\) \(dx_{\text{sim}}\). Converting to physical units:
\[
\sigma_{\text{phys}} = 11.56 \times (50 \times 10^{-10} \text{ m} / 450) \approx 1.28 \times 10^{-10} \text{ m}
\]
Comparing this to the electron's Compton wavelength, \(\lambda_C \approx 2.426 \times 10^{-12}\) m, the ratio is \(\sigma_{\text{phys}} / \lambda_C \approx 52.98\). This is a core, falsifiable prediction of EFM: the electron is an extended field structure with a characteristic size ~53 times its Compton wavelength.

\subsection{Perturbation Analysis and Energy Levels}
To probe the energy structure of the soliton, the stable ground state was perturbed and allowed to relax. The system radiated excess energy, tending towards a new stable or quasi-stable state (Figure \ref{fig:perturbed_evolution}).
\begin{figure}[htbp]
    \centering
    \includegraphics[width=\textwidth]{Perturbation.png}
    \caption{Evolution of the perturbed soliton. Max Amplitude (left) and Total Energy (right) decay from their initial perturbed values, seeking a new, lower-energy stable state.}
    \label{fig:perturbed_evolution}
\end{figure}
The energy difference between the final perturbed state and the ground state was calculated:
\begin{itemize}
    \item Ground State Energy: \( \approx 1.28 \times 10^6 \) sim. units
    \item Perturbed Final Energy: \( \approx 1.10 \times 10^7 \) sim. units
    \item Energy Difference (\(\Delta E_{\text{sim}}\)): \( \approx 9.70 \times 10^6 \) sim. units
    \item Scaled Physical Energy (\(\Delta E_{\text{eV}}\)): \( \approx 1.00 \times 10^9 \) eV (1001 MeV)
\end{itemize}
This energy gap of \(\sim 1\) GeV is on the scale of particle physics (e.g., hadron masses), not low-energy atomic transitions (e.g., Hydrogen 1s-2p at 10.2 eV). This suggests such perturbations model high-energy interactions, not simple atomic excitations.

\subsection{Fine Structure Constant (\(\alpha_{\text{em}}\)) and the Nature of Charge}
A dimensional analysis using the derived EFM scaling units allows for a calculation of the fine-structure constant. This analysis reveals that to match the observed value of \(\alpha_{\text{em,SM}} \approx 1/137\), the dimensionless charge coupling constant, \(q_{\text{sim}}\), must be calibrated to a value of \textbf{\(\approx 1.12\)}. This is a critical finding, implying that in EFM's S=T state, the fundamental charge coupling is an order-unity parameter, not a small perturbation.

\section{Discussion}
The definitive, high-resolution simulation robustly demonstrates that the EFM NLKG equation, with parameters appropriate for the S=T state (n'=1 HDS), supports the formation of a highly stable, localized eholokon. This validates EFM's core premise: that particles can be modeled as stable, self-interacting field configurations.

The model makes two profound, falsifiable predictions:
\begin{enumerate}
    \item \textbf{Particle Size}: The EFM electron analogue is an extended structure with a characteristic width (\(\sigma\)) approximately 53 times its Compton wavelength. This stands in stark contrast to the point-particle concept of the Standard Model.
    \item \textbf{The Nature of Charge Coupling}: To reproduce the known strength of electromagnetism, the dimensionless coupling parameter \(q_{\text{sim}}\) must be of order unity (\(\approx 1.12\)). This frames the EM interaction not as a weak, perturbative force, but as a fundamental, intrinsic aspect of the eholokon's structure.
\end{enumerate}
Furthermore, the perturbation analysis indicates that exciting a single eholokon corresponds to high-energy, particle-scale events (\(\sim\)GeV), reinforcing the idea that lower-energy atomic spectra must arise from the more complex interactions *between* multiple eholokons (e.g., an electron-eholokon and a proton-eholokon).

\section{Conclusion}
This computational study validates EFM's core mechanism for Higgs-less mass generation. Stable, non-rotating eholokons (solitons) form naturally within the S=T state (n'=1 HDS) of the EFM, possessing a derivable mass and a predictable, extended size. The analysis constrains EFM's dimensionless EM coupling parameter \(q_{\text{sim}}\) to be of order unity, a key insight into the nature of charge within this framework. These results provide a strong, simulation-backed foundation for EFM's particle physics, replacing postulated intrinsic properties with derived, dynamic field characteristics and offering concrete, falsifiable predictions that distinguish it from the Standard Model.

\bibliographystyle{ieeetr}
\begin{thebibliography}{99}
\raggedright
\bibitem{SMReviewPlaceholder}
Particle Data Group, et al. 2022, Prog. Theor. Exp. Phys. 2022, 083C01. 
\textit{Review of Particle Physics.}

\bibitem{larson1959}
Larson, D. B. 1959, \textit{The Structure of the Physical Universe} (Portland, OR: North Pacific Publishers).

\bibitem{emvula2025compendium_intro}
Emvula, T. 2025a, \textit{Introducing the Ehokolo Fluxon Model: A Validated Scalar Motion Framework for the Physical Universe} (Independent Frontier Science Collaboration, 2025). 

\bibitem{emvula2025efm_hds_validated} 
Emvula, T. 2025b, \textit{Foundational Validation of Eholoko Fluxon Model Harmonic Density States} (Independent Frontier Science Collaboration, May 2025).

\bibitem{emvula2025dimensionless_params} 
Emvula, T. 2025c, \textit{Dimensionless Parameters and Universal Scaling in the Ehokolo Fluxon Model}, Independent Frontier Science Collaboration, 2025.

\end{thebibliography}

\end{document}