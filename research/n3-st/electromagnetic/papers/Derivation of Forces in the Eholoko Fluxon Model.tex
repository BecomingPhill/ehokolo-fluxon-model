\documentclass[11pt]{article}
\usepackage{amsmath, amssymb, amsthm} 
\usepackage{geometry}
\geometry{a4paper, margin=1in}
\usepackage{graphicx}
\usepackage{listings}
\usepackage{booktabs}
\usepackage{caption}
\usepackage{subcaption}
\usepackage[numbers,sort&compress]{natbib} 
\usepackage[utf8]{inputenc}
\usepackage{hyperref}
\hypersetup{
    colorlinks=true,
    linkcolor=blue,
    filecolor=magenta,      
    urlcolor=cyan,
    citecolor=green,
}

\lstset{
  language=Python,
  basicstyle=\footnotesize\ttfamily,
  breaklines=true,
  numbers=left,
  numberstyle=\tiny\color{gray}, % Smaller line numbers
  commentstyle=\color{gray},
  frame=single,
  keywordstyle=\color{blue},
  stringstyle=\color{red},
  showstringspaces=false,
  tabsize=2 % Reduce tab size
}

\raggedbottom
\Urlmuskip=0mu plus 2mu\relax
\hyphenation{Eholoko-Fluxon Harmonic-Density Reciprocal-System Klein-Gordon Electro-magnetic Electro-static}
\setlength{\parskip}{0.5\baselineskip}

% --- Paper Specific Information ---
\title{Derivation of Forces in the Eholoko Fluxon Model: \\ A Unified Field Perspective}
\author{Tshuutheni Emvula\thanks{Independent Researcher, Team Lead, Independent Frontier Science Collaboration}}
\date{May 7, 2025} % Reflecting current date

\begin{document}

\maketitle

\begin{abstract}
The Standard Model (SM) describes fundamental forces through distinct gauge interactions mediated by specific bosons, requiring the Higgs mechanism for mass. The Eholoko Fluxon Model (EFM) offers a unified alternative, deriving forces from the self-interaction dynamics of a single scalar field (\(\phi\)) operating within distinct, computationally validated Harmonic Density States (HDS) (\(\rho_{n'} = \rho_{\text{ref}}/n'\)). This paper presents the computational derivation of analogues for the Electromagnetic (EM), Weak, and Strong nuclear forces within the EFM framework. Using 3D Nonlinear Klein-Gordon (NLKG) simulations (up to \(N=400^3\)) specific to the primary EFM states (S=T resonant, T/S quantum, S/T cosmic), we demonstrate: (1) EM Force Analogue (S=T state): An iterative electrostatic simulation using a complex scalar field coupled to an emergent potential \(A_0\) yields a repulsive \(1/r^2\) Coulomb-like force between charged Eholokon analogues. (2) Weak Force Analogue (T/S state): Simulations of Eholokon collisions show interaction-driven transformation and merging, consistent with the EFM mechanism for processes mediated by the T/S state dynamics, rather than spontaneous decay. (3) Strong Force Analogue (S/T state): Simulations with high nonlinearity (\(g=10\)) demonstrate a short-range attractive force leading to stable, low-amplitude bound states, mimicking nuclear binding and confinement. These results establish that force phenomena analogous to the fundamental interactions of the SM emerge deterministically from the state-dependent dynamics of the unified Eholokon field within the validated HDS structure, replacing the need for postulated gauge bosons. Code snippets and simulation parameters are provided, with full details in supplementary Jupyter Notebooks.
\end{abstract}

\section{Introduction}
The Standard Model (SM) successfully catalogues fundamental particles and their interactions but relies on separate gauge theories for the electromagnetic, weak, and strong forces, mediated by distinct bosons (photons, W/Z bosons, gluons) \citep{sm_review_placeholder}. Unification attempts within the SM are partial (electroweak), and the framework requires the Higgs mechanism for mass generation and leaves fundamental questions about particle properties and force origins unanswered. General Relativity (GR), describing gravity, remains distinct from this quantum framework.

The Eholoko Fluxon Model (EFM) proposes a unified description derived from first principles of scalar motion and space/time reciprocity (\(s \cdot t = k\)) \citep{larson1959, emvula2025compendium_intro_oct}. EFM posits a single scalar field (\(\phi\)), the Eholokon field, whose dynamics, governed by a Nonlinear Klein-Gordon (NLKG) equation, generate all physical phenomena. Crucially, EFM operates within computationally validated, discrete Harmonic Density States (HDS), \(\rho_{n'} = \rho_{\text{ref}}/n'\), which provide stable regimes for the field's primary operational states: Space/Time (S/T), Time/Space (T/S), and Space=Time (S=T) \citep{emvula2025efm_hds_validation_paper}.

In previous EFM work, it was hypothesized that fundamental forces emerge from the state-dependent interactions of stable Eholokon field structures \citep{emvula2025origin_particles_forces}. This paper presents the computational validation of this hypothesis, demonstrating how analogues of the EM, Weak, and Strong forces are derived directly from the NLKG dynamics within the appropriate EFM state (S=T, T/S, S/T respectively). We show results from 3D simulations that reproduce characteristic force laws and interaction behaviors without postulating gauge bosons, offering a deterministic, unified alternative grounded in field dynamics.

\section{Mathematical Framework: Forces from Field Dynamics}
The EFM framework utilizes variants of the NLKG equation tailored to specific states and interactions. The baseline equation incorporating potential self-interaction and coupling terms is:
\begin{equation}
\frac{\partial^2 \phi}{\partial t^2} - c_{\text{eff}}^2 \nabla^2 \phi + V'(\phi) + [\text{State/Interaction Terms}] = 0
\label{eq:nlkg_general}
\end{equation}
where \(V'(\phi) = m^2 \phi - g \phi^3 + \eta \phi^5\). The state/interaction terms vary depending on the force being modelled. All dynamics occur within the validated HDS structure (\(\rho_{n'} = \rho_{\text{ref}}/n'\)).

\subsection{Electromagnetic Force Analogue (S=T State)}
Modelled using a complex scalar field \(\phi\) coupled to the electromagnetic potential \(A_\mu\) via the covariant derivative \(D_\mu = \partial_\mu - i q A_\mu\). The relevant Lagrangian (simplified) is \( \mathcal{L} = \frac{1}{2}|D_\mu \phi|^2 - V(|\phi|) - \frac{1}{4}F_{\mu\nu}F^{\mu\nu} \). In the electrostatic approximation for static Eholokons (\(\phi(\mathbf{x},t) = f(r)e^{i\omega t}\), \(A_\mu = (A_0(\mathbf{x}), 0, 0, 0)\)), this leads to coupled equations:
\begin{align}
[-\nabla^2 + m^2 - g|\phi|^2 + \eta|\phi|^4 - (\omega - qA_0)^2] \phi &= 0 \label{eq:nlkg_static_em} \\
\nabla^2 A_0 &= -J_0 \approx -2q\omega|\phi|^2 \label{eq:poisson}
\end{align}
These are solved iteratively to find self-consistent \(\phi\) and \(A_0\) fields, from which the interaction force is derived. This simulation operates in the S=T state (hypothesized \(\alpha=1.0, c_{\text{eff}}=1\)) within a relevant density level (e.g., \(n'=3\)).

\subsection{Weak Force Analogue (T/S State)}
Modelled as interaction-driven transformations governed by the baseline NLKG (Eq. \ref{eq:nlkg_baseline}) with T/S state parameters: reduced effective speed (\(c_{\text{eff}}^2 = 0.1\)), state parameter (\(\alpha=0.1\)), and operating within the \(n'=2\) HDS level. No explicit decay term is added; transformations arise from the collision dynamics themselves. Dissipation (\(\delta\)) can be included but was found ineffective or destabilizing in preliminary tests for simple decay.
\begin{equation}
\frac{\partial^2 \phi}{\partial t^2} - (0.1 c^2) \nabla^2 \phi + m^2 \phi - g \phi^3 + \eta \phi^5 = 0 \quad (\text{T/S State, No Dissipation})
\label{eq:nlkg_ts}
\end{equation}

\subsection{Strong Force Analogue (S/T State)}
Modelled as binding arising from strong self-interaction in the baseline NLKG (Eq. \ref{eq:nlkg_baseline}) within the S/T state parameters: standard effective speed (\(c_{\text{eff}}^2 = 1.0\)), state parameter (\(\alpha=0.1\)), and crucially, a **large cubic nonlinearity \(g \gg 0.1\)** (e.g., \(g=10.0\)). This operates within a low-\(n'\) HDS level associated with cosmic structure/binding.
\begin{equation}
\frac{\partial^2 \phi}{\partial t^2} - c^2 \nabla^2 \phi + m^2 \phi - \mathbf{g_{\text{strong}}} \phi^3 + \eta \phi^5 = 0 \quad (\text{S/T State, Strong Coupling})
\label{eq:nlkg_st}
\end{equation}

\section{Simulation Results: Deriving Force Analogues}
Simulations were performed using PyTorch on \(N=400^3\) grids, employing methods detailed in the HDS validation paper \citep{emvula2025efm_hds_validation_paper} and the supplementary notebooks.

\subsection{Electromagnetic Force (S=T State)}
The iterative electrostatic simulation for two charged Eholokons (using complex \(\phi\), Eq. \ref{eq:nlkg_static_em} \& \ref{eq:poisson}) converged to stable solutions for various separations \(d\).
\begin{figure}[htbp]
    \centering
    \includegraphics[width=\textwidth]{emforce_N400_energy_force.png}
    \caption{Interaction energy (left) and derived force (right) vs. separation \(d\) for two charged Eholokons in the S=T state (N=400). Positive energy and force indicate repulsion. Dashed lines show fits to \(A/r+C\) (energy) and \(B/r^2+D\) (force), consistent with Coulomb's law (\(B \approx 5.59\times 10^1 > 0\)).}
    \label{fig:em_force_energy_force}
\end{figure}
\begin{figure}[htbp]
    \centering
    \includegraphics[width=0.9\textwidth]{emforce_N400_final_fields_sep7.3.png}
    \caption{Final field slices (\(z=0\)) for separation \(d \approx 7.33\): Magnitude \(|\phi|\) (left) shows two distinct, stable Eholokons. Electrostatic potential \(A_0\) (right) shows positive peaks at Eholokons causing repulsion.}
    \label{fig:em_force_final_fields}
\end{figure}
Figure \ref{fig:em_force_energy_force} shows a positive interaction energy that decreases with separation, indicating repulsion. The derived force is positive and fits well to a \(1/r^2\) law, successfully reproducing the Coulomb force analogue. Figure \ref{fig:em_force_final_fields} shows the stable Eholokon structures and the generated repulsive potential \(A_0\). (Simulation details in `forceelectrostatic.ipynb`).

\subsection{Weak Force Interaction (T/S State)}
The simulation of two Eholokons colliding in the T/S state (Eq. \ref{eq:nlkg_ts}) showed interaction leading to transformation, not simple decay.
\begin{figure}[htbp]
    \centering
    \includegraphics[width=\textwidth]{ts_collision_N400_stability_metrics.png}
    \caption{Stability metrics for T/S Eholokon collision (N=400). Max amplitude (left) and average density (center) remain bounded after interaction. Total energy (right) is conserved after the initial interaction/merging phase.}
    \label{fig:ts_collision_metrics}
\end{figure}
\begin{figure}[htbp]
    \centering
    \includegraphics[width=0.5\textwidth]{ts_collision_N400_final_state.png}
    \caption{Final field slice (\(z=0\)) after T/S collision (N=400). The initial two structures have interacted and merged into a single, stable central structure.}
    \label{fig:ts_collision_final_state}
\end{figure}
Figure \ref{fig:ts_collision_metrics} shows the system stabilizing after the collision, with conserved energy. Figure \ref{fig:ts_collision_final_state} reveals the initial structures merged into a single, stable entity. This demonstrates transformation via interaction mediated by T/S dynamics, consistent with the EFM concept for the Weak Force analogue. (Simulation details in `ts_collision.ipynb`).

\subsection{Strong Force Binding (S/T State)}
The simulation of two Eholokons in the S/T state with strong coupling (\(g=10.0\), Eq. \ref{eq:nlkg_st}) resulted in a stable bound state.
\begin{figure}[htbp]
    \centering
    \includegraphics[width=\textwidth]{strongforce_N400_energy_force.png}
    \caption{Interaction energy (left) and derived force (right) vs. separation \(d\) for two Eholokons in the S/T state with \(g=10\) (N=400). The energy shows a clear attractive well (\(E_{int} < 0\)). The force shows short-range repulsion and longer-range attraction, crossing zero at the energy minimum (\(d \approx 8.4\)). Yukawa-like fits capture the qualitative behavior.}
    \label{fig:strong_force_energy_force}
\end{figure}
\begin{figure}[htbp]
    \centering
    \includegraphics[width=0.6\textwidth]{strongforce_N400_final_field_sep8.4.png}
    \caption{Final field slice (\(z=0\)) at the energy minimum (\(d \approx 8.44\)) for the S/T strong coupling simulation (N=400). Shows a tightly bound, low-amplitude, complex configuration.}
    \label{fig:strong_force_final_field}
\end{figure}
Figure \ref{fig:strong_force_energy_force} clearly shows an attractive energy well (\(E_{int} < 0\)) and a force characteristic of nuclear binding (short-range repulsion, longer-range attraction). Figure \ref{fig:strong_force_final_field} shows the low-amplitude, potentially merged configuration of the bound state, suggesting confinement. (Simulation details likely in `force.ipynb` or `force2ehos.ipynb`).

\section{Discussion}
The simulations presented provide strong computational evidence that analogues of the three fundamental forces described by the Standard Model emerge naturally from the state-dependent dynamics of the single scalar Eholokon field (\(\phi\)) within the EFM framework.
\begin{itemize}
    \item The **Electromagnetic Force** analogue arises from interactions mediated by the complex \(\phi\) field coupled to an emergent electrostatic potential \(A_0\) within the resonant **S=T state**, correctly reproducing the repulsive \(1/r^2\) Coulomb law for like charges.
    \item The **Weak Force** analogue is represented by interaction-driven transformations and relaxations within the dynamic **T/S state**. Collisions lead to merging and energy redistribution, providing a mechanism for particle transformations without spontaneous decay of isolated stable Eholokons in the baseline model.
    \item The **Strong Force** analogue emerges from the high nonlinearity (\(g\)) of the \(\phi\) field interactions within the stable **S/T state**, leading to short-range attraction, binding, and potentially confinement-like behavior in low-amplitude merged structures.
\end{itemize}
Crucially, these distinct force characteristics arise not from postulated gauge bosons but from the different dynamical behaviors intrinsic to the S/T, T/S, and S=T states, which themselves operate within the validated Harmonic Density State structure (\(\rho_{n'} \propto 1/n'\)). This provides a unified, deterministic, and mechanistic origin for fundamental forces.

\section{Conclusion}
The Eholoko Fluxon Model successfully derives analogues for the electromagnetic, weak, and strong forces from the state-dependent dynamics of a single scalar field (\(\phi\)) operating within its validated Harmonic Density State framework. 3D computational simulations demonstrate: (1) Coulomb-like \(1/r^2\) repulsion via an emergent potential in the S=T state; (2) Interaction-driven transformation/merging in the T/S state; (3) Short-range binding via strong nonlinearity in the S/T state. These results validate the core EFM hypothesis that fundamental forces are emergent properties of Eholokon field dynamics, replacing the need for SM gauge bosons and offering a unified, first-principles foundation for particle interactions.

\appendix
\section{Core Simulation Logic Snippets}
\label{app:code}
Conceptual Python/PyTorch code snippets illustrating the core logic for each force simulation are provided below. Full implementations are in the supplementary Jupyter Notebooks (`forceelectrostatic.ipynb`, `ts_collision.ipynb`, `force.ipynb`/`force2ehos.ipynb`).

\subsection{EM Force (Iterative Electrostatic Solver)}
\begin{lstlisting}[caption=EM Force Logic, label=lst:em_code]
# Setup complex phi, real A0
phi = create_initial_guess(sep, ...) # Complex tensor
A0 = torch.zeros_like(phi.real) 
E_single = calculate_single_energy(...)

for iter_num in range(max_iterations):
    phi_old = phi.detach().clone()
    # Solve static NLKG for phi, including A0 term
    phi, _ = solve_nlkg_static(phi_old.requires_grad_(True), A0, ...) 
    # Calculate charge density J0 ~ |phi|^2
    J0 = (2.0 * q * omega * torch.abs(phi)**2).to(A0.dtype)
    # Solve Poisson for A0: lap(A0) = -J0
    A0 = solve_poisson_fft(J0, ...) 
    # Check convergence: torch.norm(phi - phi_old) / torch.norm(phi_old)
    if converged: break
# Calculate interaction energy: E_total - 2 * E_single
E_int, E_tot = calculate_interaction(phi, A0, ..., E_single) 
\end{lstlisting}

\subsection{Weak Interaction (T/S Collision)}
\begin{lstlisting}[caption=T/S Collision Logic, label=lst:ts_code]
# Setup initial phi1, phi2, phi=phi1+phi2
# Setup initial phi_dot based on velocity
phi = create_initial_guess(...) # Real tensor here
phi_dot = calculate_initial_velocity(...) # Real

for t in range(T_steps):
    # Evolve using RK4 with T/S parameters (low c_eff, alpha=0.1, delta=0)
    phi, phi_dot = update_phi_rk4_tsc(phi, phi_dot, dt, ...) 
    # Monitor metrics (max_amp, avg_rho, energy)
# Analyze final state structure and energy conservation
\end{lstlisting}

\subsection{Strong Binding (S/T State, High g)}
\begin{lstlisting}[caption=Strong Force Logic, label=lst:st_code]
# Setup initial phi1, phi2, phi=phi1+phi2 (real or complex ok)
# Parameter g is set to a large value (e.g., 10.0)
# Use static solver (like EM force, but A0=0, q=0) for different separations
E_single = calculate_single_energy_strong(...) 
interaction_energies = []
for sep in separations:
     phi_guess = create_initial_guess(sep, ...)
     # Solve static NLKG for phi (no A0 term)
     phi_final, E_total = solve_nlkg_static(phi_guess, A0_zero, ...) 
     E_int = E_total - 2 * E_single
     interaction_energies.append(E_int)
# Calculate Force F = -dE_int/d(sep)
\end{lstlisting}

\bibliographystyle{ieeetr}
\begin{thebibliography}{99}
\raggedright

% --- Core EFM / Foundational ---
\bibitem{larson1959}
Larson, D. B. 1959, \textit{The Structure of the Physical Universe} (Portland, OR: North Pacific Publishers).

\bibitem{emvula2025compendium_intro_oct} % General Intro / Template
Emvula, T. 2025a, \textit{Introducing the Eholoko Fluxon Model: A Scalar Motion Framework for the Physical Universe} (Independent Frontier Science Collaboration, October 2025). 

\bibitem{emvula2025rst} % RST Math Framework
Emvula, T. 2025b, \textit{A Mathematical Framework for the Reciprocal System Theory} (Independent Frontier Science Collaboration, February 2025).

\bibitem{emvula2025efm_hds_validation_paper} % HDS Validation Paper (Drafted)
Emvula, T. 2025c, \textit{Foundational Validation of Eholoko Fluxon Model Harmonic Density States} (Independent Frontier Science Collaboration, May 2025).

\bibitem{emvula2025origin_particles_forces} % Key Context Paper
Emvula, T. 2025d, \textit{The Origin of Particles and Forces: A Unified Derivation from the Eholoko Fluxon Model} (Independent Frontier Science Collaboration, April 13, 2025).

\bibitem{emvula2025efm_qft_unification} % QFT/Force Unification Concept Paper
Emvula, T. 2025e, \textit{Eholoko Fluxon Model: Eholokon Quantum Field Theory and Force Unification} (IFSC, Mar 16, 2025).

% --- Specific Phenomena / Validation Papers ---
\bibitem{emvula2025lagrangian_validation} % Relevant for EM section
Independent Frontier Science Collaboration. 2025, \textit{Fluxonic Lagrangian Validation: Numerical and Theoretical Analysis} (IFSC, February 27, 2025).

\bibitem{emvula2025matter_formation} % Combined or choose one of the Mar 16 papers
Emvula, T. 2025f, \textit{Eholoko Fluxon Model: Eholokon Matter Formation Across Atomic, Molecular, and Macroscopic Scales} (IFSC, Mar 16, 2025). 
% Note: If the two Mar 16 matter papers are distinct, add the other one too.

% --- Standard Model / External Data References (Keep as needed) ---
\bibitem{sm_review_placeholder} 
Particle Data Group, Zyla, P. A., et al. 2020, Prog. Theor. Exp. Phys., 2020, 083C01. 
\textit{Review of Particle Physics.}

\end{thebibliography}

\end{document}